\section*{Введение}
\addcontentsline{toc}{section}{\hspace{7mm}Введение}
На современные компьютеры необходимо устанавливать сотни программ в рамках
одной операционной системы. В операционной системе Linux программное 
обеспечение распространяется в форме пакетов, тесно связанных друг с другом,
которые находятся в централизованном хранилище в сети, --- репозитории. По
сути, не предполагается возможности получения программного обеспечения из 
других источников, таким образом, репозиторий представляет собой хранилище 
колоссальных размеров, где пакеты тесно связаны друг с другом.

Основным свойством репозитория является целостность, что подразумевает
возможность установки пользователем любого пакета, в нем находящегося. 
Обеспечение этого свойства составляет одну из основных задач утилит для 
работы с репозиторием. 

В компании  ALT Linux было принято решение о создании новой системы
управления пакетам  - пакетного менеджера Deepsolver, поэтому возникла 
необходимость проверки целостности репозитория данной системы, по сути, 
качества работы упомянутых утилит. 

Таким образом, целью настоящей работы является создание программного 
продукта, проверяющего целостностноть репозитория Deepsolver.
