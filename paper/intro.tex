\section*{Введение}
\addcontentsline{toc}{section}{\hspace{7mm}Введение}

На современные компьютеры необходимо устанавливать сотни программ в рамках
одной операционной системы. В операционной системе GNU/Linux программное 
обеспечение (ПО) распространяется в форме пакетов, тесно связанных друг с другом,
которые находятся в централизованном хранилище в сети, --- репозитории. 
Фактически, не предполагается возможности получения ПО из 
других источников, таким образом, репозиторий представляет собой хранилище 
колоссальных размеров, где пакеты тесно связаны друг с другом.

Основным свойством репозитория является целостность, что подразумевает
возможность установки пользователем любого пакета, в нем находящегося. 
Обеспечение этого свойства составляет одну из основных задач утилит для 
работы с репозиторием. 

В компании  ALT~Linux было принято решение о создании новой системы
управления пакетами  - пакетного менеджера Deepsolver, поэтому возникла 
необходимость проверки целостности репозитория данной системы, по сути, 
качества работы упомянутых утилит. 

Таким образом, целью настоящей работы является проектирование 
инструментария контроля целостности репозитория, удовлетворяющего условиям эксплуатации в~окружении ALT~Linux, 
и реализация на~базе подготовленного проекта необходимых утилит.
