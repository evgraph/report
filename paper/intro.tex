\section*{Введение}
\addcontentsline{toc}{section}{\hspace{7mm}Введение}
На соврменные компьютеры необходимо устанавливать сотни программ внутри
одной операционной системы. В операционной системе Linux программное 
обеспечение распространяется в форме пакетов, тесно связанных друг с другом,
которые находятся в централизованном хранилище в сети - репозитории. По
сути, не предполагается возможности получения программного обеспечения из 
других источников, таким образом, репозиторий представляет собой хранилище 
колоссальных размеров, а пакеты имеют множестве связей друг с другом.

Для корректной работы репозитория необходимо поддержание его актуальности, 
иначе говоря, обеспечение возможности установки пользователем любого пакета,
находящегося в репозитории. Для решения этой задачи необходимо осуществлять
проверку в двух направлениях: 
\begin{itemize}
\item{Проверка текущего состояния репозитория}
\item{Проверка механизма обновления репозитория}
\end{itemize}

Цель данной работы состоит в реализации утилиты для решения приведенных
выше задач для репозитория пакетного мендежера Deepsolver от компании
ALT Linux.









