
\documentclass[a4paper]{article}

\usepackage[russian]{babel}
\usepackage[utf8]{inputenc}
\usepackage{cmap}
\usepackage{amsmath}
\usepackage{amssymb}
\usepackage{xspace}
\usepackage{graphicx}
\usepackage{sectsty}
\usepackage{fancyhdr}
\usepackage{longtable}
\usepackage{url}
\usepackage{appendix}
\usepackage{enumerate}

\newcommand{\EN}[1]{{#1}}
\newcommand{\CODE}[1]{{\ttfamily #1}}

\fancyhf{}
\fancyfoot[C]{\Large \thepage}
\renewcommand{\headrulewidth}{0pt}
\renewcommand{\footrulewidth}{0pt}

%%Modifying captions for figures and tables:
\makeatletter
\long\def\@makecaption#1#2{%
\vspace{\abovecaptionskip}%
\sbox{\@tempboxa}{\Large #1.~#2}
\ifdim \wd\@tempboxa >\hsize
{\Large #1.~#2}\par
\else
\global\@minipagefalse
\hbox to \hsize {\hfill {\Large #1.~#2}\hfill}%
\fi
\vspace{\belowcaptionskip}}

\renewcommand{\baselinestretch}{1.2}
%%\sectionfont{\LARGE}
\sectionfont{\Large}
\subsectionfont{\Large}
\subsubsectionfont{\Large}
\setlength{\belowcaptionskip}{6pt}
\makeatletter \renewcommand{\@biblabel}[1]{#1.\hfill}

\makeatletter 
\def\redeflsection{\def\l@section{\@dottedtocline{1}{1.5em}{7.8em}}} 
\renewcommand\appendix{\par 
\setcounter{section}{0}% 
\setcounter{subsection}{0}% 
\def\@chapapp{\appendixname}% 
\addtocontents{toc}{\protect\redeflsection} 
\def\thesection{\appendixname\hspace{0.2cm}\@Asbuk\c@section}} 
\makeatother 

\textwidth = 17cm
\oddsidemargin= 0 pt
\topmargin = -1cm
\headheight = 0cm
\headsep = 0cm
\textheight = 26.5cm

\begin{document}
\huge

Здравствуйте!
Меня зовут Кирюшкина Валентина. Тема моей выпускной квалификационной
работы - "Создание утилит для проверки целостности репозитория пакет
ного менедежра Deepsolver". Мой научный руководитель - Пожидаев
Михаил Сергеевич.

\newpage
На современные компьютеры необходимо устанавливать сотни программ.
В операционной системе Линукс существует своя собственная модель 
для дистрибуции ПО, центральным элементом которой является репозиторий 
- хранилище программ, находящееся в сети. Основным свойством репозитория
является целостность, что подразумевает возможность установки 
пользователем любого ПО, в нем находящегося. Обеспечение этого 
свойства составляет одну из основных задач утилит для работы с 
репозиторием. 

В ос Линукс программа представляет собой пакет - файл, содержащий 
информацию о продукте, программные файлы и документацию. 
Пакеты внутри репозитория имеют множество связей с другими пакетами, 
называемых зависимостями. 
\newpage
В компании  ALT Linux было принято решение о создании новой системы
управления пакетам  - пакетного менеджера Deepsolver, поэтому возникла 
необходимость контроля целостности репозитория данной системы, по сути, 
качество работы упомянутых утилит. %%лучше скаазать
\newpage
Целью настоящей работы является создание программного 
продукта, проверяющего целостностноть репозитория Deepsolver.
\newpage



 Пакет может предоставлять функциональность другого пакета, указывая об 
этом информацию в тэге provide. 
Так же допустимы следующие типы отношений на множестве пакетов:
\begin{itemize}
\item
Requires : пакет требует обязательное наличие другого пакета, указанного 
по его имени или по одному из его provides.
\item
Conflicts :
пакет запрещает наличие другого пакета, указанного по его имени или по
provides.
\item 
Obsoletes: Пакет может указывать, что является обновлением некоторого 
множества пакетов. 
\end{itemize}

\newpage

В репозитории Deepsolver содержится около 60 тысяч пакетов, в среднем у каждого 
из них по 9 require и по 2 provide.
Таким образом, очевидно, что обеспечение целостности при добавлении или удалении пакета - 
нетривиальная задача, которая может повлечь за собой множественные изменения.


Эти изменения отражаются в метаданных репозитория - индексе.
\newpage
В  Deepsolver реализован специальный механизм для внесения изменений в индекс,
являющийся частью автоматизированной сборочницы пакетного менеджера, 
предназначенной для сборки пакетов. Некорректная работа данного механизма влечет за собой 
нарушение целостности индексной информации. Поэтому важным этапом в решении 
проблемы обеспечения актуального состояния индекса является контроль правильности 
работы данного механизма. 
\newpage
Таким образом, цели настоящей работы могут быть определены, как:
\begin{itemize}
\item
Создание утилиты для проверки готового индекса.
\item
Тестирование утилит для обновления индекса.
\end{itemize}


Помимо обнаружения критических ошибок, немаловажно отслеживать появление 
неактуальных данных, засчет которых неоправданно увеличивается размер индекса. %Добавить к целям?
\newpage
Индекс репозитория Deepsolver хранится на ftp-сервере ALT Linux, представляет
из себя набор файлов, содержащих информацию различной детализации
для каждого типа архитектуры.
\newpage
Прежде чем перейти к описанию реализованных утилит стоит
определить понятие целостности репозитория. Под целостностью
репозитория понимается состояние, которое удовлетворяет следующим
условиям:
\begin{itemize}
\item{Для каждого \textit{require} пакета существует одноименный \textit{provide} 
другого пакета. Require, для которого это условие не выполняется
называется \textit{анметом (unmet)}. }
\item{Для каждого \textit{provide} существует одноименный \textit{require/conflict} или
он находится в директории из заранее заданного списка.Это условие 
контролирует появление избыточных \textit{provide}, засчет которых
может существенно расти размер индекса.}
\item{Для каждого \textit{conflict} пакета существует одноименный \textit{provide} 
другого пакета. Это условие не является таким строгим, как два предыдущих,
но желательно, чтобы оно выполнялось. }
\end{itemize}
\newpage
Утилита для проверки готового индекса на вход получает индекс-файл, содержащий основной
список пакетов с информацией о 
зависимостях между ними. После чего проводит сбор информации о пакетах и их зависимостях. 
На основании полученных данных првоеряется, удовлетворяет ли текущее состояние 
репозитория условиям целостности. В случае отрицательного ответа выводится 
статистика о найденных ошибках. Обнаружение серьезных ошибок и недоточетов 
на первом этапе тетсирования является индикатором некорректной работы 
механизма обновления индекса. В зависимости от степени серьезности обнаруженных
повреждений, администратор репозитория принимает решение о запуске второго этапа 
тестирования или временном игнорировании.
\newpage
В качестве языка для реализации решения был выбран Python. На данном слайде 
представлена структура модуля, предоставляющего основную функциональность 
для проведения проверки. Объекты класса Package представляют отдельные пакеты 
и содержат информацию о пакете, полученную из индекса: имя пакета и множества 
зависимостей.

Класс IndexParser отвечает за разбор файлов индекса. Основной метод класса 
- \textit{get\_packages\_from\_index\_file} - возвращает генератор массива 
обхектов класса Package, передавая в конструкторе значения полей класса, 
полученных в результате разбора.генератор - поддерживаемая на синтаксическом уровне 
реализация итератора --- типового решения для инкапсуляции последовательного 
доступа к элементам объекта-агрегата. 
Тот факт, что метод \textit{get\_packages\_from\_index\_file} возвращает генератор позволяет
его использовать в том числе и для парсинга большого количества файлов большого размера, 
таким образом, что все множество пакетов необязательно должно сохраняться в память ---
при каждой новой итерации будет прочитан следующий один блок файла индекса, 
а содержание предыдущего может быть удалено из ОЗУ.

Класс IndexTestResult предназначен для удобной работы с результатами тестирования 
и их выводом. У класса определен метод diff, получающий на вход объект класса IndexTestResult. 
Результатом его работы является изменение объекта, его вызвавшего, в соответствии
с объектом, полученным в качестве аргумента.

В классе IndexTester определны методы, непосредственно отвечающие за проверку индекса.
\newpage
Целью второго этапа тестирования является обнаружение неполадок в механизме 
обновления. На вход, поступает индекс репозитория. Сначала осуществляется 
проверка на неизмененном индексе. После чего запускаются
утилиты для обновления индекса, причем после каждой итерации осуществляется 
проверка измененного индекса. Выводится статистика для каждой проверки, 
на основании которой администратор может более точно локализовать ошибку.


\newpage
Такимо образом, в результате данной ВКР быи разработаны:\\
Утилита для проверки готового индекса\\
Утилита для проверки механизма обновления инжекса.\\
\\
Разработанные утилиты активно использовались на стадии отладки процедуры обновления
индекса Deepsolver, с их помощью проводился контроль за влиянием изменений. На дан-
ный момент индекс Deepsolver интегрирован в официальный репозиторий компании ALT Linux.
Его проверка на целостность была проведена с использованием реализованных утилит и дала
положительный результат.


\end{document}









 




