
\documentclass[a4paper]{article}

\usepackage[russian]{babel}
\usepackage[utf8]{inputenc}
\usepackage{cmap}
\usepackage{amsmath}
\usepackage{amssymb}
\usepackage{xspace}
\usepackage{graphicx}
\usepackage{sectsty}
\usepackage{fancyhdr}
\usepackage{longtable}
\usepackage{url}
\usepackage{appendix}
\usepackage{enumerate}

\newcommand{\EN}[1]{{#1}}
\newcommand{\CODE}[1]{{\ttfamily #1}}

\fancyhf{}
\fancyfoot[C]{\Large \thepage}
\renewcommand{\headrulewidth}{0pt}
\renewcommand{\footrulewidth}{0pt}

%%Modifying captions for figures and tables:
\makeatletter
\long\def\@makecaption#1#2{%
\vspace{\abovecaptionskip}%
\sbox{\@tempboxa}{\Large #1.~#2}
\ifdim \wd\@tempboxa >\hsize
{\Large #1.~#2}\par
\else
\global\@minipagefalse
\hbox to \hsize {\hfill {\Large #1.~#2}\hfill}%
\fi
\vspace{\belowcaptionskip}}

\renewcommand{\baselinestretch}{1.2}
%%\sectionfont{\LARGE}
\sectionfont{\Large}
\subsectionfont{\Large}
\subsubsectionfont{\Large}
\setlength{\belowcaptionskip}{6pt}
\makeatletter \renewcommand{\@biblabel}[1]{#1.\hfill}

\makeatletter 
\def\redeflsection{\def\l@section{\@dottedtocline{1}{1.5em}{7.8em}}} 
\renewcommand\appendix{\par 
\setcounter{section}{0}% 
\setcounter{subsection}{0}% 
\def\@chapapp{\appendixname}% 
\addtocontents{toc}{\protect\redeflsection} 
\def\thesection{\appendixname\hspace{0.2cm}\@Asbuk\c@section}} 
\makeatother 

\textwidth = 17cm
\oddsidemargin= 0 pt
\topmargin = -1cm
\headheight = 0cm
\headsep = 0cm
\textheight = 26.5cm

\begin{document}
\huge

Здравствуйте!
Меня зовут Кирюшкина Валентина. Тема моей выпускной квалификационной
работы - "Создание инструментария для тестирования целостности репозитория 
пакетного менедежра Deepsolver". Мой научный руководитель - Пожидаев
Михаил Сергеевич.

\newpage
На современные компьютеры необходимо устанавливать сотни программ.
В операционной системе Линукс существует своя собственная модель 
для дистрибуции ПО, центральным элементом которой является репозиторий 
- хранилище программ, находящееся в сети. Основным свойством репозитория
является целостность, что подразумевает возможность установки 
пользователем любого ПО, в нем находящегося. Обеспечение этого 
свойства составляет одну из основных задач утилит для работы с 
репозиторием. 

В ос Линукс программа представляет собой пакет - файл, содержащий 
информацию о продукте, программные файлы и документацию. 
Пакеты внутри репозитория имеют множество связей с другими пакетами, 
называемых зависимостями. 
\newpage
В компании  ALT Linux было принято решение о создании новой системы
управления пакетам  - пакетного менеджера Deepsolver, поэтому возникла 
необходимость контроля целостности репозитория данной системы, по сути, 
качества работы упомянутых утилит. %%лучше сказать
\newpage
Таким образом, целью настоящей работы является проектирование 
инструментария контроля целостности репозитория, удовлетворяющего условиям эксплуатации в~окружении ALT~Linux, 
и реализация на~базе подготовленного проекта необходимых утилит
\newpage



 Пакет может предоставлять функциональность другого пакета, указывая об 
этом информацию в тэге provide. 
Так же допустимы следующие типы отношений на множестве пакетов:
\begin{itemize}
\item
Requires : пакет требует обязательное наличие другого пакета, указанного 
по его имени или по одному из его provides.
\item
Conflicts :
пакет запрещает наличие другого пакета, указанного по его имени или по
provides.
\item 
Obsoletes: Пакет может указывать, что является обновлением некоторого 
множества пакетов. 
\end{itemize}

\newpage

В репозитории Deepsolver содержится около 60 тысяч пакетов, в среднем у каждого 
из них по 9 require и по 2 provide.
Таким образом, очевидно, что обеспечение целостности при добавлении или удалении пакета - 
нетривиальная задача, которая может повлечь за собой множественные изменения.


Эти изменения отражаются в метаданных репозитория - индексе.
\newpage
В  Deepsolver реализован специальный механизм для внесения изменений в индекс,
являющийся частью автоматизированной сборочницы пакетного менеджера, 
предназначенной для сборки пакетов. Некорректная работа данного механизма влечет за собой 
нарушение целостности индексной информации. Поэтому важным этапом в решении 
проблемы обеспечения актуального состояния индекса является контроль правильности 
работы данного механизма. 
\newpage
Таким образом, цели настоящей работы могут быть определены, как:
\begin{itemize}
\item
Создание утилиты для проверки готового индекса.
\item
Тестирование утилит для обновления индекса.
\end{itemize}


Помимо обнаружения критических ошибок, немаловажно отслеживать появление 
неактуальных данных, засчет которых неоправданно увеличивается размер индекса. %Добавить к целям?
\newpage
Индекс репозитория Deepsolver хранится на ftp-сервере ALT Linux, представляет
из себя набор файлов, содержащих информацию различной детализации
для каждого типа архитектуры.
\newpage
Прежде чем перейти к описанию реализованных утилит стоит
определить понятие целостности репозитория. Под целостностью
репозитория понимается состояние, которое удовлетворяет следующим
условиям:
\begin{itemize}
\item{Для каждого \textit{require} пакета существует одноименный \textit{provide} 
другого пакета.  }
\item{Для каждого \textit{provide} существует одноименный \textit{require/conflict} или
он находится в директории из заранее заданного списка.}
\item{Для каждого \textit{conflict} пакета существует одноименный \textit{provide} 
другого пакета. }
\end{itemize}
Таким образом, проверка на целостность сводится к проверки выполнения вышеописанных
условий.
\newpage
При нарушении условий целостности возможны следующие ошибки:
\begin{itemize}
\item
require, не удовлетворяющие первому условию называются анметами.
\item
provide, не удовлтворяющие второму условию считаются избыточными.
\item
conflict, не удовлетворяющие третьему условию, не являются индикатором
серьезной ошибки.
\end{itemize}
Анметы - это серьезные ошибки, нарушающие целостность индекса. Избыточные
provide не влияют на работу индекса, но неоправданно увеличивают его размер.

\newpage

Утилита для проверки готового индекса на вход получает индекс-файл, содержащий основной
список пакетов с информацией о 
зависимостях между ними. После чего проводит сбор информации о пакетах и их зависимостях. 
На основании полученных данных првоеряется, удовлетворяет ли текущее состояние 
репозитория условиям целостности. В случае отрицательного ответа выводится 
статистика о найденных ошибках. Обнаружение серьезных ошибок(анметов) и недоточетов 
на первом этапе тетсирования является индикатором некорректной работы 
механизма обновления индекса. В зависимости от степени серьезности обнаруженных
повреждений, администратор репозитория принимает решение о запуске второго этапа 
тестирования или временном игнорировании.
\newpage
Статистика об ошибках выводит информацию об обнаруженных нарушениях целостности и имеет следующий формат: количество найденных пакетов, общее количество пакетов, содержащий ошибки, количество пакетов, содержащий каждый вид ошибок. Так же по запросу пользователя может
выводится подробная статистика о найденных повреждениях.
\newpage
В качестве языка для реализации решения был выбран Python. На данном слайде 
представлена диаграмма классов модуля, предоставляющего основную функциональность 
для проведения проверки. Объекты класса Package представляют отдельные пакеты 
и содержат информацию о пакете, полученную из индекса: имя пакета и множества 
зависимостей.

Класс IndexParser отвечает за разбор файлов индекса.


Класс IndexTestResult предназначен для удобной работы с результатами тестирования 
и их выводом. 

В классе IndexTester определны методы, непосредственно отвечающие за проверку индекса.

 Основной метод класса IndexParser 
- \textit{get\_packages\_from\_index\_file} - возвращает генератор массива 
обхектов класса Package, передавая в конструкторе значения полей класса, 
полученных в результате разбора.Генератор - поддерживаемая на синтаксическом уровне 
реализация итератора --- типового решения для инкапсуляции последовательного 
доступа к элементам объекта-агрегата. 
Тот факт, что метод \textit{get\_packages\_from\_index\_file} возвращает генератор позволяет
его использовать в том числе и для парсинга большого количества файлов большого размера, 
таким образом, что все множество пакетов необязательно должно сохраняться в память ---
при каждой новой итерации будет прочитан следующий один блок файла индекса, 
а содержание предыдущего может быть удалено из ОЗУ.
\newpage

Таким образом, первый этап проверки сводится к выполнению следующих действий:

\begin{enumerate}
\item
После запуска утилиты создается объект класса \textit{IndexParser} и вызывается его метод
\textit{get\_packages\_from\_index\_directory()}, куда в качестве выбранной
директории передается строка, полученная в качестве аргумента утилиты.

\item
В качестве результата работы метод \textit{get\_packages\_from\_index\_directory()}
возвращается объект-генератор, элементами которого являются отдельные пакеты из индекса 
репозитория.

\item
После этого создается объект класса \textit{IndexTester} и вызывается его метод
\textit{test\_index()} для объекта-генератора, полученного на предыдущем шаге.

\item
Перед началом работы метода \textit{test\_index()} происходит сбор всех существующих
в индексе версий пакетов и их provides-строк. Полученное множество строк сохраняется
в виде хэш-таблицы для обеспечения оптимального времени поиска.

\item
Таким образом, для определения наличия повреждений целостности
необходимо проверить, что для каждого из requires и provides всех пакетов
существует соответствующее вхождение в хэш-таблице из предыдущего шага.

\item 
Для каждого найденного повреждения необходимо добавить его в объект \textit{IndexTestResult}.

\item
После чего пользователю выводится подробная статистическая информация о найденных повреждениях.

\end{enumerate}
.
\newpage
Целью второго этапа тестирования является обнаружение неполадок в механизме 
обновления. На вход, поступает индекс репозитория. Сначала осуществляется 
проверка на неизмененном индексе. После чего запускаются
утилиты для обновления индекса, причем после каждой итерации осуществляется 
проверка измененного индекса. Выводится статистика для каждой проверки, 
на основании которой администратор может более точно локализовать ошибку.
Статистика имеет формат вывода ранее описанной утилиты.


\newpage
Такимо образом, в результате данной ВКР быи разработаны:\\
Утилита для проверки готового индекса\\
Утилита для проверки механизма обновления инжекса.\\
\\
Реализация и внедрение утилит производилась в~несколько этапов. 
Первоначальная реализация отрабатывалась на~индексной информации, созданной в~отладочных целях вне~окружения сборочной системы ALT~Linux.
Далее производилось внедрение поддержки Deepsolver на~сборочной площадке,
в~ходе которой корректность работы контролировалась при~помощи утилит, описанных в~настоящем отчёте.
Финальная часть работ была завершена 20~мая 2013~г.
В~результате полная поддержка индексной информации Deepsolver была развёрнута на~сборочном окружении, 
находящегося в~промышленной эксплуатации.
Окончательный результат был успешно протестирован автором работы
без~выявления каких-либо неисправностей и недостатков.


\end{document}









 




