\section{Репозиторий}
В операционных система Linux программное обеспеченье распространяется
с помощью репозиториев - специальных хранилищ, чаще всего располагающихся
на серверах в Инетернете. Сама программа представляет собой файл, содержащий
информацию о продукте, программные файлы и документацию, - пакет. Таким образом,
если сформулировать иначе, репозиторий содержит набор пакетов. Помимо этого
в репозитории содержится вспомогателньая информация о данных пакетах - индекс.
Для корректного добавления, удаления и обновления пакетов, а так же их инсталляции
используются специальные программы, называемые пакетными менеджерами. 

В настоящей работе речь пойдет о пакетном менеджере Deepsolver, поэтому описанное
ниже необходимо воспринимать относительно именно этого пакетного менеджера.
%%FIXME:кривовато звучит

\section{ Пакеты }
Deepsolver работает с пакетами формата RPM. RPM  - рекурсивный акроним RPM Package Manager,
ранее раскрывался, как Red Hat Package Manager - обозначает две сущности: формат пакетов и 
связанная с ним программа.
По сути, сущствует два основных вида форматов пакетов: 
\begin{itemize}
\item{Пакеты, включающие исходные коды программ.}
\item{Бинарные пакеты - пакеты, содержащие исполняемые файлы.}
\end{itemize}
RPM относится ко второму виду.

Название rpm-пакета состоит из нескольких частей и в общем виде может быть представлено
такой строкой: \\
name-version-release.architecture.rpm , где:\\
\begin{itemize}
\item{name - имя пакета (чаще всего - название основного приложения, содержащегося в пакете);}
\item{version - версия}
\item{release - релиз}
\item{architecture - архитектура}
\end{itemize}

\subsection {Отношения на множестве пакетов}
Пакет может предоставлять функциональность других пакетов, указывая
информацию об этом в тэге provides. Имя provides допускает произвольное
значение, не обязательно совпадающее с именем какого-либо существующего
 пакета и является, скорее, соглашением, что пакет обладает некоторой
совместимостью. Для provides допускается указание подмножества версии.
Неявными provides считаются имена всех файлов, хранимых в пакете.

Следующие типы отношений допускаются на множестве пакетов:
\begin{itemize}
\item{\b Requires: пакет требует обязательное наличие другого пакета, указанного
по его имени или по одному из его provides. Допускается указание 
подмножества версии требуемого пакета. В случае указания ограничения версии
под requires может подходить provides только дополненный информацией
о версии.}
\item{Conflicts: пакет запрещает наличие другого пакета, указанного по его имени
 или по одному из его provides. Допускается указание подмножества
версии конфликтуемого пакета. В случае указания ограничения версии
под conflicts может подходить provides только дополненный информацией
о версии.}
\item{Obsoletes: Пакет может указывать, что является обновлением некоторого
множества пакетов. При установке такого пакета все пакеты, обновлением
которых он является, удаляются из ОС. Попытка их установки после
установки обновляющего пакета приводит к ошибке типа “установлена
более свежая версия”. Допускается указание подмножества версии обновляемых
пакетов. В случае указания ограничения версии под obsoletes может
подходить provides только дополненный информацией о версии.}
\end{itemize}
%%ССЫЛКА:http://deepsolver.altlinux.org/tech-assignment.pdf

\section{Сборка пакетов}
В репозитории Deepsolver-а существует специальная программа для сборки пакетов. Работает
она следующим образом: автор пакета присылает ссылку на git-репозиторий с исходным кодом, %%сделать ссылку про git
после чего в строго автоматизированном порядке программа выполняет сборку пакета и кладет
его в репозиторий. Сборка может осуществляться как перед помещением пакета в репозиторий,
так и в любое другое время, если администратор решил пересобрать репозиторий.
%%FIXME: можно дописать про girar

\section{Задача администрирования Deepsolver}
Deepsolver для правильной работы требует выполнения некоторых задач 
администрирования, которые включают в себя задачи обслуживания репозиториев
 пакетов и настройку Deepsolver на рабочих местах.

\subsection{Обслуживание  репозиториев}
Все задачи администрирования репозиториев сводятся к созданию и поддержке в
актуальном состоянии специальной вспомогательной информации, называемой 
“индексом репозитория”. Индекс используется для хранения подробной информации 
о наборе пакетов в репозитории, он доставляется в первую очередь на компьютеры
пользователей, и от его актуальности зависит корректность обработки запросов 
на внесение изменений в состояние ОС. В простейшем варианте индекс
хранит перечисление доступных пакетов с необходимыми атрибутами, включая
полный список provides, из-за чего размер файлов индекса становится
чрезмерно большим. Тем не менее, на практике доля provides, действительно
задействованных в вычислении зависимостей между пакетами, невелика,
и это даёт возможность для различных оптимизаций размера хранимой информации.

Правильный подход настройки фильтрации provides выбирается исходя
из назначения репозитория. Существуют следующие режимы:
\begin{itemize}
\item{1. Фильтрация на основе поиска соответствующих записей requires/conflicts.
Если репозиторий является единственным источником пакетов для установки
на компьютеры пользователей, и нет других сторонних источников ПО, 
то эффективное сокращение количества provides может быть
достигнуто за счёт включения режима фильтрации на основе известных
requires/conflicts. При активации этого режима запись provides сохраняется
в индексе только в том случае, если известна хотя бы одна запись
requires или conflicts, в которой имя пакета совпадает с именем пакета
в provides. Информация о версии в этом случае не учитывается. В терминологии 
Deepsolver соответствующие записи requires/conflicts иногда называются “ссылками”.}
\item{2. Фильтрация на основе каталогов файлов. Режим фильтрации provides
на основе информации о каталоге файла может применяться только для
тех записей provides, которые были сгенерированы автоматически на основе
 списка файлов пакета. Deepsolver имеет возможность указания списка
каталогов, файлы из которых всегда обрабатываются как допустимые
provides. Эта возможность полезна для выполнения запросов на установку
 пакетов по имени файлов. Например, по имени файлов из каталога
/usr/bin.}
\end{itemize}
Если ни один из указанных режимов фильтрации provides не используется,
то индекс содержит все возможные provides. Если указаны оба из них, то
в индекс попадает provides, если он подходит хотя бы под одно правило.

Deepsolver предоставляет возможность внесения изменений в индекс без
полной перегенерации, что порождает ряд трудностей при использовании
фильтрации provides на основе известных requires/conflicts. При появлении
новых пакетов нарушается общая целостность, поскольку корректное добав-
ление требует восстановления ранее исключённых provides. Описанные ниже
утилиты позволяют правильно обновлять содержимое индекса и поддерживать
его в целостном состоянии.
%%ССЫЛКА: Deepsolver user manual

\subsection{Утилиты для работы с репозиторием}
Для решение вышеописанных проблем используются утилиты ds-repo, ds-patch,
ds-provides, являющиееся частью упомянутой ранее программы для сборки пакетов.
В качестве одного из аргументов каждая из них требует путь к директории с файлами индекса.

\subsubsection{ds-repo}
 ds-repo выполняет все действия по созданию индекса для некоторой
одной компоненты репозитория. Главными параметрами для неё являются
путь к каталогу, где должен быть расположен новый индекс, и множество
путей к каталогам с пакетами, которые должны быть зарегистрированы. 
Целевой каталог создаётся автоматически. Если он уже существует, то не 
должен содержать других каталогов или файлов. Каталоги с пакетами могут
быть перечислены в произвольном порядке и могут содержать как бинарные
пакеты, так и пакеты с исходными текстами. Утилита ds-repo автоматически
распознает их тип и помещает в соответствующий раздел индекса. Имена
всех описанных каталогов указываются как свободные параметры командной
строки. На первом месте должен быть указан целевой каталог, за которым
следует перечисление каталогов с пакетами. Пропуск целевого каталога
не допускается. Если ни один из каталогов с пакетами не указан, то 
просматривается текущий каталог, установленный на момент запуска утилиты.

\subsubsection{ds-patch}
Утилита ds-patch производит включение и исключение файлов из построенного
индекса Deepsolver. Все параметры, указываемые при вызове ds-repo,
повторно перечислять не нужно. Они сохраняются в информационном файле
индекса репозитория и загружаются автоматически при изменении его содержимого.
При запуске ds-patch производится проверка контрольных сумм, и
если регистрируется факт обнаружения повреждённого файла, работа завершается
с ошибкой. После работы суммы изменённых файлов обновляются.
При использовании утилиты ds-patch необходимо помнить, что утилита
только изменяет набор файлов, перечисленных в индексе, но не обраба-
тывает множество provides, которое после изменения теряет целостность и
должно быть отдельно обработано утилитой ds-provides, описанной ниже. 
Целостность списка provides нарушается не только в изменённом индексе, но и
во всех других взаимосвязанных компонентах репозитория. Таким образом,
для них также должна быть произведена обработка утилитой ds-provides.
Утилита ds-patch ожидает указания одного свободного параметра — каталога
с индексом для изменения. Если он не указан, то используется текущий
каталог. Файлы для включения или исключения должны перечисляться после ключей
- -add и - -del соответственно. Файлы для включения должны быть перечислены 
с указанием абсолютного пути к файлу пакета, в то время как файлы для удаления 
перечисляются просто по своему имени. Как и утилита ds-repo, утилита ds-patch
производит автоматическое определение типа пакета, и регистрирует его в
соответствующем разделе индекса. Для отделения указания целевого каталога 
от предшествующего перечисления списка файлов необходимо использовать
последовательность“- -”.

\subsubsection{ds-provides}
Утилита ds-provides производит исправление множества provides в индексе
некоторой компоненты репозитория после внесения изменений в неё или
в связанные с ней компоненты. Производить запуск этой утилиты требуется
только в том случае, если выбран режим фильтрации provides на основе
списка известных requires/con?icts. В качестве свободного параметра при 
вызове указывается путь к каталогу с файлами индекса. Если он отсутствует,
то используется текущий каталог, установленный на момент вызова. В начале
работы утилита ds-provides проверяет контрольные суммы, и если они
не совпадают, то завершает работу аварийно. После выполнения всех 
необходимых действий файл с контрольными суммами обновляется. В качестве
дополнительного параметра может быть указан параметр - -ref-sources,
назначение которого полностью аналогично назначению одноимённого параметра
для утилиты ds-repo.



