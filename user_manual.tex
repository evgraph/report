\section{Руководство пользователя}
Утилиты для проверки индекса и для проверки утилит для обновления
запускаются из командной строки. 
Утилита для проверки индекса имеет следующий формат запуска: 
$ds_index_test.py [-h] [-all] [-u] [-p] [-c] DIRECTORY_FILE INDEX_FILE [INDEX_FILE ...]]$

По ключу $-h выводится справка об утилите.
DIRECTORY_FILE - аргумент, соответствующий текстовому файлу (*.txt) с 
директориями, provide из которых не будет считаться избыточным. Директории
перечисляются  через двоеточие. Пример содержимого файла:
$/bin:/usr/bin:/usr/local/bin:/usr/X11R6/bin:/usr/games$

INDEX_FILE - файл формата $*.data.gz$, содержащий информацию о пакетах
в формате описанном в %%FIXME:сделать ссылку. Таких файлов может быть несколько,
и они перечисляются через пробел.

Результат теста представляет собой статистику, содержащую информацию 
о количестве найденных и поврежденных пакетов, количество пакетов содержащих
каждый вид повреждений:
\begin{itemize}
\item{избыточный provide}
\item{анмет - require, для которого не существует соответствующего provide}
\item{избыточный conflict},
а так же подробную информацию о том, какой пакет какое повреждение содержит.
Результат работы выводится в консоль и имеет следующий формат:
Found [количество найденных пакетов] packages
[количество поврежденных пакетов] packages are damaged
[количество пакетов с избыточными provide] packages have unmatched provides
[количество пакетов с анметами] packages have unmets
[количество пакетов с избыточными conflict] packages have unmatched conflicts
Подробная информация о повреждениях имеет следующий формат:
package [имя пакета] has unmatched [тип: provide или conflict] [строка соответствующая повреждению]
package [имя пакета] has [тип: unmet] [строка соответствующая повреждению]

Вывод подробной информации можно регулировать с помощью ключей:
-all --- при указании этой опции выводятся все повреждения(избыточные require и conflict, unmets и
соответствующие им пакеты),
-u --- при указании этой опции выводятся только unmets и соответствующие им пакеты,
-p --- при указании этой опции выводятся только  избыточные provides,
-c --- при указании этой опции выводятся только избыточные conflicts.

Утилита для проверки утилит обновления индекса имеет следующий формат запуска:
$ds_patch_util_test.py [-h] [-all] [-u] [-p] [-c] DIRECTORY_FILE INDEX_FILES_PATH [INDEX_FILES_PATH ...]$, где
позиционные аргументы:
DIRECTORY_FILE - аргумент, соответствующий текстовому файлу (*.txt) с 
директориями, provide из которых не будет считаться избыточным,
INDEX_FILES_PATH - путь до директории с индекс-файлами: 
\begin{itemize}
\item{\textit{info} --- информационный файл с параметрами индекса;} 
\item{\textit{rpms.complete.data) --- вспомогательный файл, не предназначенный
для загрузки пользователями, с информацией для повторной фильтрации
provides;}
\item{\textit{rpms.data} --- основной список пакетов с информацией о зависимостях между ними;}
\item{\textit{rpms.descr.data} --- список пакетов с расширенными описаниями;}
\item{\textit{rpms.filelist.data} --- списки файлов для каждого бинарного пакета;}
\item{\textit{srpms.data} --- основная информация о пакетах с исходными текстами;}
\item{\textit{srpms.descr.data} --- список пакетов с исходными текстами, содержащий
расширенную информацию.}
\end{itemize}.
Опциональные аргументы определяются аналогично предыдущему случаю. 
В данной утилите результат будет иметь  формат, подобный описанному ранее, 
только блоки подобного формата будут выводиться после каждого удаления пакета.


